\documentclass{report}

\begin{document}
  \begin{center}
  \textbf{\huge{Rapport Projet Maths}}
  \end{center}
  \begin{enumerate}
  \item
    Le Lalalala :
  \begin{equation}
    \partial_t
    \left( \begin{array}{c}
    \rho \\
    \rho u \\
    Q \\
    B
    \end{array}
  \right)
  + \nabla .
  \left( \begin{array}{c}
    \rho u \\
    \rho u \otimes u + (p + \frac{B.B}{2})I - B \otimes B \\
    (Q + p + \frac{B.B}{2})u - (B.u)B \\
    u \otimes B - B \otimes u \\
    \end{array}
  \right)
  = 0,
  \end{equation}
  \begin{equation}
    Q = \rho e + \rho \frac{u.u}{2} + \frac{B.B}{2}
  \end{equation}
  \begin{equation}
    p = P(\rho,e) = (\gamma - 1)\rho e, \gamma > 1.
  \end{equation} \\


  \item
   En utilisant la m\'ethode propos\'ee en 2002, ce systeme devient : \\
  \begin{center}
    \begin{math}
      \frac{\partial}{\partial t}W + \sum\limits_{i=1}^d \frac{\partial}{\partial x_i}F^i(W) = 0.
    \end{math}
  \end{center}

  En consid\'erant :
  \begin{math}
    \partial\star = \frac{\partial}{\partial x_\star}
  \end{math}
   et
   \begin{math}
     \partial_iF^i(W) = \sum\limits_{i=1}^d\frac{\partial}{\partial x_i}F^i(W).
   \end{math} \\
   On obtient l'\'equation :
   \begin{math}
     \partial_tW+\partial_iF^i(W) = 0.
   \end{math}
 \end{enumerate}
\end{document}
